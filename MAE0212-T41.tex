\documentclass{article}

\usepackage[portuguese]{babel}
\usepackage[utf8]{inputenc}
\usepackage{indentfirst}
\pagestyle{empty}
\date{%2\textordmasculine\ Semestre de 2011}
	}
\title{Avalição de Ensino\\
       MAE0212\\
       Introdução à Probabilidade e à Estatística II}

\author{}
\addtolength{\topmargin}{-1in}
\textheight = 695pt





\begin{document}

\maketitle
\thispagestyle{empty}

A avaliação desta disciplina, de modo geral, foi a melhor de todas. A didática do professor foi considerada ``muito boa'' por 75\% da classe, e ``boa'' pelo restante. A relação professor-estudante é respeitada, as aulas contribuem bastante para o aprendizado dos alunos, o material didático recomendado \cite{FMC} \cite{FPA} é muito bom e condizente com a explicação do professor, e a maioria dos alunos considerou ``bom'' o próprio comprometimento. A relação entre o conteúdo ensinado e a sua possível aplicação prática é bem explorada, e isso é \emph{muito} útil, no BCC, para motivar os alunos a interessar-se por disciplinas não puramente ``algorítmicas''.\\
\indent A frequência em aula - dos 36 alunos que lá estavam - é de aproximadamente 86\%, o que nos leva a concluir que os alunos que de fato vão às aulas, vão com bastante frequência. Alunos que já cursaram esta disciplina elogiaram a melhoria da explicação do professor, e o caracterizaram como ``mais interessado com os alunos''. O comprometimento dos alunos, assim como o seu aprendizado, foi considerado ``bom''. Contudo, o estudo se concentra nas semanas de provas, durante as quais percebemos surgir uma necessidade por aulas de exercícios e revisão de conceitos.\\
\indent Como único ponto discutível da disciplina está a estrutura das avaliações. Percebemos, pelos comentários dos alunos, que a primeira prova tinha poucas questões, mas que estas eram longas e muito específicas. Alguns conceitos que os alunos julgavam ser importantes (estruturamento dos algoritmos de fatoração, ou provas de propriedades de normas) - e que por isso foram mais extensamente estudados - não foram cobrados. Deste modo, as notas foram menores do que as esperadas. Além disso, alguns alunos consideraram a correção da disciplina muito rigorosa. 

\vspace{6ex}

João Marco Maciel da Silva

\emph{joao.marco.silva@usp.br}

%\vspace{11ex}

%\begin{thebibliography}{99}



%\end{thebibliography}


\end{document}
       
       
       
