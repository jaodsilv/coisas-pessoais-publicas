%%%%%%%%%%%%%%%%%%%%%%%%%%%%%%%%%%%%%%%%%
% Plasmati Graduate CV
% LaTeX Template
% Version 1.0 (24/3/13)
%
% This template has been downloaded from:
% http://www.LaTeXTemplates.com
%
% Original author:
% Alessandro Plasmati (alessandro.plasmati@gmail.com)
%
% License:
% CC BY-NC-SA 3.0 (http://creativecommons.org/licenses/by-nc-sa/3.0/)
%
% Important note:
% This template needs to be compiled with XeLaTeX.
% The main document font is called Fontin and can be downloaded for free
% from here: http://www.exljbris.com/fontin.html
%
%%%%%%%%%%%%%%%%%%%%%%%%%%%%%%%%%%%%%%%%%


\documentclass[a4paper,10pt]{article} % Default font size and paper size

\usepackage{fontspec} % For loading fonts

\defaultfontfeatures{Mapping=tex-text}
%\setmainfont[SmallCapsFont = Verdana SmallCaps]{Verdana} % Main document font

\usepackage{xunicode,xltxtra,url,parskip} % Formatting packages

\usepackage[usenames,dvipsnames]{xcolor} % Required for specifying custom colors

%\usepackage[big]{layaureo} % Margin formatting of the A4 page, an alternative to layaureo can be 
\usepackage{fullpage}
% To reduce the height of the top margin uncomment: \addtolength{\voffset}{-1.3cm}

\usepackage{hyperref} % Required for adding links	and customizing them
\definecolor{linkcolour}{rgb}{0,0.2,0.6} % Link color
\hypersetup{colorlinks,breaklinks,urlcolor=linkcolour,linkcolor=linkcolour} % Set link colors throughout the document

\usepackage{titlesec} % Used to customize the \section command
\titleformat{\section}{\Large\scshape\raggedright}{}{0em}{}[\titlerule] % Text formatting of sections
\titlespacing{\section}{0pt}{3pt}{3pt} % Spacing around sections

\begin{document}

\pagestyle{empty} % Removes page numbering

\font\fb=''[cmr10]'' % Change the font of the \LaTeX command under the skills section

%----------------------------------------------------------------------------------------
%	NAME AND CONTACT INFORMATION
%----------------------------------------------------------------------------------------

\par{\centering{\Huge \textsc{João Marco Maciel da Silva}}\bigskip\par} % Your name

\section{Dados Pessoais}

\begin{tabular}{rl}
\textsc{Local e Data de Nascimento:} & Brasília-DF, Brasil | 25 de Maio de 1988 \\
\textsc{Endereço:} & Av. Eng. Heitor Antonio Eiras Garcia, 125. apt. 54.\\& Vila Pirajussara. São Paulo-SP, Brasil \\
\textsc{Telefone:} & +55 11 98737 0300\\
\textsc{E-mail:} & \href{mailto:joaodasisilva@gmail.com}{joaodasisilva@gmail.com}\\
& \href{mailto:jaodsilv@linux.ime.usp.br}{jaodsilv@linux.ime.usp.br}\\
\textsc{Página Pessoal:} & \href{http://linux.ime.usp.br/jaodsilv}{http://linux.ime.usp.br/jaodsilv}\\
\end{tabular}


%----------------------------------------------------------------------------------------
%	WORK EXPERIENCE 
%----------------------------------------------------------------------------------------

\section{Experiência}

\begin{tabular}{r|p{11cm}}

\textsc{Set 2010-Jan 2013} & Desenvolvedor no \textbf{CCSL}, São Paulo-SP, Brasil \emph{}\\
& \footnotesize{Linguagens: Java, Ruby}\\
       & \footnotesize{Desenvolvedor nos projetos Kalibro e Mezuro do CCSL. Pesquisa em métricas de código em
       sistemas \emph{open source}}
\multicolumn{2}{c}{} \\

%------------------------------------------------

\textsc{Jan 2010-Jul 2010} & IT member at \textbf{RedeCASD}, São José dos Campos-SP, Brasil \emph{}\\
& \footnotesize{Linguagens: Bash Script, C, C++}\\
       & \footnotesize{Operação, desenvolvimento e manutenção da rede de computadores dos estudantes do ITA(Instituto
       Tecnológico de Aeronáutica/SP, Brasil), a rede era composta por cerca de 7 servidores Ubuntu e cerca
       de 600 computadores com diferentes SO(Windows, MacOS, UNIX, Linux e outros). Trabalho Voluntário.}
       
\multicolumn{2}{c}{} \\

%------------------------------------------------

\textsc{Jan 2009-Jul 2010} & Membro de conselho do \textbf{DepCult}, São José dos Campos-SP, Brasil \emph{}\\
& \footnotesize{Membro de Conselho do \textit{Departamento Cultural} do 
       \emph{Centro Acadêmico Santos Dummont}. Trabalho Voluntário.}
\multicolumn{2}{c}{} \\


%------------------------------------------------

\textsc{Jun 2008-Jan 2009} & Diretor Financeiro do \textbf{DepCult}, São José dos Campos-SP, Brasil \emph{}\\
& \footnotesize{Diretor financeiro no \emph{Departamento Cultural} do 
       \emph{Centro Acadêmico Santos Dummont}. Trabalho Voluntário.}
\multicolumn{2}{c}{} \\


%------------------------------------------------

\textsc{Jan 2008-Jan 2009} & Membro do \textbf{DepCult}, São José dos Campos-SP, Brasil \emph{}\\
& \footnotesize{Logistica em eventos culturais do \emph{Departamento Cultural} do \emph{Centro Acadêmico Santos Dummont}. Trabalho voluntário.}

\end{tabular}


%----------------------------------------------------------------------------------------
%	EDUCATION
%----------------------------------------------------------------------------------------

\section{Formação}

\begin{tabular}{r|p{11cm}}
\textsc{Fev} 2011-\textsc{Dez} 2014& Bacharelado em \textsc{}\textsc{Ciência da Computação} \\& \normalsize\textbf{Universidade de São Paulo}, São Paulo-SP, Brasil\\
\multicolumn{2}{c}{}\\

%------------------------------------------------

\textsc{Fev} 2013-\textsc{Jul} 2013 & Semestre de Intercâmbio\\& \textbf{Università della Svizzera Italiana}, Lugano-TI, Suíça\\
\multicolumn{2}{c}{} \\
    
%------------------------------------------------

\textsc{Jan} 2003-\textsc{Dez} 2005& Ensino médio comum\\&\textbf{GGE Colégio e Curso}, Recife-PE, Brasil\\
\end{tabular}

\section{Outros Cursos}

\begin{tabular}{r|p{11cm}}

2011 & \textbf{K19 Treinamentos}, S\~ao Paulo-SP, Brasil\\
    & \footnotesize{Treinamento \emph{Desenvolvimento Web com JSF2 e JPA}}\\
    & \footnotesize{Treinamento \emph{Persist\^{e}ncia com JPA2 e Hibernate}}\\
    & \footnotesize{Treinamento \emph{Desenvolvimento Web Avançado com JSF2, EJB3.1 e CDI}}\\
    & \footnotesize{Treinamento \emph{Integração de Sistemas com Webservices, JMS e EJB}}\multicolumn{2}{c}{} \\

2009 & \textbf{Caelum Treinamentos}, S\~{a}o Paulo-SP, Brasil\\
    & \footnotesize{Treinamento \emph{Java e Orientção a Objetos}}\\
    & \footnotesize{Treinamento \emph{Laboratório Java com Testes, XML e Design Patterns}}\\
    & \footnotesize{Treinamento \emph{Java para Desenvolvimento Web}}\multicolumn{2}{c}{} \\

2009 & \textbf{CPORAer}, São José dos Campos-SP, Brasil\\
    & \footnotesize{Curso Preparatório de Oficiais da Reserva da Aeronáutica}\\

\end{tabular}

%----------------------------------------------------------------------------------------
%	LANGUAGES
%----------------------------------------------------------------------------------------

\section{Línguas}

\begin{tabular}{rl}
\textsc{Inglês:} & Fluente\\

\textsc{Português:} & Língua materna\\

\textsc{Italiano:} & Conhecimento básico\\
\end{tabular}

%----------------------------------------------------------------------------------------
%	COMPUTER SKILLS 
%----------------------------------------------------------------------------------------

\section{Conhecimentos Técnicos}

\begin{tabular}{r|p{11cm}}
Linguagens: & Java, C, C++, Ruby, Scala, Erlang, HTML,\\& JavaScript, Bash, Python\multicolumn{2}{c}{} \\

Application Servers e Web Servers: & JBoss AS, TomCat\multicolumn{2}{c}{} \\

Frameworks, Ferramentas e Especificações Java: & JSF2, Hibernate, Struts2, JUnit, Maven,\\& Ant, JPA2, EJB3.1, JMX, JMS, Swing, LOG4J,\\& JDBC, JSP, JSTL, JAX-WS, JAX-RS, JTA, JTS\multicolumn{2}{c}{} \\

Outros Frameworks: & Rails, Savon, RUnit, Cucumber, Lift\multicolumn{2}{c}{} \\

Banco de Dados: & MySQL\multicolumn{2}{c}{} \\

Sistemas Operacionais: & Todas as Distribuições Linux \\&Debian-Based, ArchLinux, Slackware, \\&Windows 98/XP/Vista/7, MAC OS X\\& Mountain Lion.\\


\end{tabular}

%----------------------------------------------------------------------------------------
%	EVENTS CONGRESS CONFERENCES
%----------------------------------------------------------------------------------------

\section{Eventos, Congressos e Conferências}

\begin{tabular}{rl}
\textsc{Jul} 2012 & \textbf{FISL13} - Palestrante\\

\textsc{Nov} 2011 & \textbf{RubyConf Brasil 2011} - Espectador\\
\textsc{Out} 2011 & \textbf{International Conference On Open Source Systems 2011} - Apresentação de Poster\\
\textsc{Set} 2011 & \textbf{MiniPLOP Brasil 2011} - Participante

\end{tabular}

%----------------------------------------------------------------------------------------
%	INTERESTS AND ACTIVITIES
%----------------------------------------------------------------------------------------

\section{Interesses e Atividades}

Problemas de Olimpíada de Matemática, tecnologia, leitura, fotografia, música, artes,\\
boxe, artes marciais, beisebol, futebol, aprender.

%----------------------------------------------------------------------------------------

\end{document}
