\documentclass{article}

\usepackage[portuguese]{babel}
\usepackage[utf8]{inputenc}
\usepackage{indentfirst}
\pagestyle{empty}
\date{}
\title{Avalição de Ensino\\
       MAT0139\\
       Álgebra Linear para Computação}
       
\author{}
\addtolength{\topmargin}{-1in}
\textheight = 695pt


\begin{document}

\maketitle
\thispagestyle{empty}

\indent Sobre o conteúdo dado e cobrado em Álgebra Linear para Computação, lecionada pelo professor \textit{Claudio 
Gorodski} não foram feitas críticas.\\
\indent Os elogios mais frequentes foram relacionados as listas de exercícios e as aulas de exercícios, que, em geral,
foram citadas como boas, coerentes com o cobrado em avaliações. \\
\indent Devido a alguns descontentamentos em relação ao andamento da aula, a frequência de presença em aula está abaixo 
do esperado. Apesar de o comprometimento dos alunos dentro e fora da sala ser regular, houve reclamações quanto à 
didática do professor, afirmando-se que as aulas pouco tem colaborado para o aprendizado, e quanto ao material didático, 
pois muitos o consultam apenas por o professor se basear apenas nele.\\
\indent O professor e os alunos poderiam se empenhar mais para deixar a aula mais produtiva, com os alunos perguntando 
mais durante a aula e aumentando o tempo de estudo fora dela(lembrando de dividir esse tempo com as demais matériasque 
são feitas), e o professor esclarecendo as dúvidas mais cuidadosamente.


\vspace{6ex}

João Marco Maciel da Silva

\emph{joao.marco.silva@usp.br}

\vspace{11ex}

\begin{thebibliography}{99}

	\bibitem{STR} \emph{Álgebra Linear e suas aplicações}, G. Strang.
	
\end{thebibliography}

\end{document}
