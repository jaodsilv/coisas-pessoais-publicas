\documentclass{article}

\usepackage[portuguese]{babel}
\usepackage[utf8]{inputenc}
\usepackage{indentfirst}
\pagestyle{empty}
\date{}
\title{Avaliação de Ensino\\
       MAT0139\\
       Álgebra Linear para Computação}
       
\author{}
\addtolength{\topmargin}{-1in}
\textheight = 695pt


\begin{document}

\maketitle
\thispagestyle{empty}

\indent Sobre o conteúdo dado e cobrado em \textit{Álgebra Linear para Computação}, lecionada pelo professor 
\textit{Claudio Gorodski} não foram feitas críticas e parece estar coerente com o programa.\\
\indent Os elogios mais frequentes foram relacionados as listas de exercícios e as aulas de exercícios que, em geral,
foram citadas como boas ou ótimas e coerentes com o cobrado em avaliações. \\
\indent As critícas relacionadas ao método de ensino e a aula em si foram muitas, os alunos reclamaram principalmente
 com relação a voz baixa nas explicações, a aparente não preparação da aula. Foi citado também que o professor se 
 perde nos exemplos, próprios ou do livro, servindo como argumento para dizer que o professor não prepara a aula.
  Foi dito também que o professor não explica claramente apesar de ser percebido boa vontade neste ponto.\\
\indent Quanto ao material didático houveram apenas críticas, no qual o professor se orienta para dar a aula, o 
~\cite{STR}, foi citado que há apenas um exemplar do livro na biblioteca, e que o livro é prolixo em suas explicações.
Também que as bibliografias alternativas sugeridas pelo professor não cobrem todo o conteúdo dado em aula e que o 
~\cite{STR} cobre.


\vspace{6ex}

João Marco Maciel da Silva

\emph{joao.marco.silva@usp.br}

\vspace{11ex}

\begin{thebibliography}{99}

	\bibitem{STR} \emph{Álgebra Linear e suas aplicações}, G. Strang.
	
\end{thebibliography}

\end{document}
